\documentclass{article}

     \usepackage[top=2cm,bottom=2cm,left=2cm,right=2cm]{geometry}
     \usepackage{amsfonts,amsmath,amssymb}
     \usepackage{color}
     \usepackage{url}
     \usepackage{tikz}
     \usetikzlibrary{patterns,shapes}

     \begin{document}

     \title{Solving RRSP$(F)$ when $F$ belongs to an interval}
     \author{Julien Khamphousone, Fabi\'an Casta\~no, Andr\'e Rossi, Sonia Toubaline}
     \date{2024-04-13}
     \maketitle
     \def\F{{0.0,1.0,1.1}}
\def\gF{{12.78,13.47,1281.66}}
\begin{figure}[ht!]
    \begin{center}
    \begin{tikzpicture}[scale=.2cm, xscale=2cm]
    \tikzset{axe/.style={->,>=stealth,draw,color=black}}
    \tikzset{g/.style={draw,color=red,line width=1pt}}
    \tikzset{convX/.style={fill=yellow}}
    \draw[axe](0,0) -- (0.25,0);
     \draw[axe](0,0) -- (0,14.817000000000002);\draw (\F[0],0) -- (\F[0],-.2);
        \draw (\F[0],0pt) -- (\F[0],-1pt) node[anchor=north,inner sep = 8pt] {\scriptsize{$F_0$}};
        \draw[thick, dashed](\F[0],0) -- (\F[0], \gF[0]);
\draw (\F[1],0) -- (\F[1],-.2);
        \draw (\F[1],0pt) -- (\F[1],-1pt) node[anchor=north,inner sep = 8pt] {\scriptsize{$F_1$}};
        \draw[thick, dashed](\F[1],0) -- (\F[1], \gF[1]);
\draw (0, \gF[0]) -- (-.2, \gF[0]);
        \draw (0, \gF[0]) -- (-1pt, \gF[0]) node[anchor=east,inner sep = 8pt] {\scriptsize{$g(F_0)$}};
        \draw[thick, dashed](0,\gF[0]) -- (\F[0],\gF[0]);
\draw (0, \gF[1]) -- (-.2, \gF[1]);
        \draw (0, \gF[1]) -- (-1pt, \gF[1]) node[anchor=east,inner sep = 8pt] {\scriptsize{$g(F_1)$}};
        \draw[thick, dashed](0,\gF[1]) -- (\F[1],\gF[1]);
\node (F) at (2.25,2){}; \draw(F) node{$F$};
    \node (g) at (2,14.817000000000002){}; \draw(g) node{$g(F)$};
    % Lines
\draw[g] (\F[0], \gF[0]) -- (\F[1], \gF[1]);% Integer points
    %\draw[fill=black] (1,2) circle(.05);
    \end{tikzpicture}
    \end{center}
    \vspace*{-2eM}
    \caption{$g(F)$ for the eil51 instance}\label{fig:1}
    \end{figure}
    Here are the $F$, $B$ and $K$ values rounded at 5 digits:

\begin{itemize}
	\item  $F = [0.0,100.0]$
 \item $B = [0.69,0.66]$
 \item  $K = [1278.0,1281.0]$\item Total execution time is: 161.1s\item Number of times ILP is called: 3\end{itemize}\end{document}